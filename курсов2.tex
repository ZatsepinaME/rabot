

%\documentclass[a4paper,russian]{article}
%\usepackage[T2A]{fontenc}
%\usepackage[T2A]{fontenc}
%\usepackage[utf8x]{inputenc}
%\usepackage{pscyr}
%\usepackage[english, russian]{babel}




\documentclass[russian,utf8,nocolumnxxxi,nocolumnxxxii]{eskdtext}
	
\usepackage[T1,T2A]{fontenc}
\usepackage[utf8]{inputenc}
%\usepackage[utf8x]{inputenc}
%\usepackage[english,ukrainian,russian]{babel}
\usepackage{amssymb,amsmath}

\usepackage{tikz}
\usepackage{siunitx}
\usepackage[american,cuteinductors,smartlabels]{circuitikz}


\usepackage{indentfirst} % Красная строка
%\usepackage[backend=biber]{biblatex}
%\addbibresource{error_estimation_otchet.bib}
\mathtoolsset{showonlyrefs=true} % Показывать номера только у тех формул, на которые есть \eqref{} в тексте.
\usepackage[]{hyperref}
\hypersetup{
	colorlinks=true,
}
\usepackage{multirow}
\usepackage{textcomp}
\newcommand{\No}{\textnumero}

\ESKDdepartment{Федеральное агентство по образованию}
\ESKDcompany{Санкт-Петербургский государственный электротехнический университет "ЛЭТИ"}
\ESKDtitle{Пояснительная записка по дисциплине "Информатика"}
\ESKDdocName{Вариант №7}
	\ESKDsignature{Курсовая работа}
	\ESKDauthor{Сhhhhhhнт ~А.~А.}
	\ESKDchecker{Прокшин~А.~Н.}
	
   \begin{document}
   %	\maketitle
 	
    \maketitle
    \tableofcontents 
    \newpage
     \section{ Вступление}
    \itshape{\Large Цель курсовой работы:}\\[4pt]
    \rmfamily{уметь применять персональный компьютер и математические пакеты прикладных программ в инженерной деятельности}\\[18pt]
    ${\,\,\,\,\,\,\,\,\,\,\,\,\,\,\,\,\,}$\itshape{\Large Тема курсовой работы:}\\[4pt]
    \rmfamily{решение математических задач с использованием математического пакета "Scilab" или "Reduce-algebra".}\\
    
    \newpage
    \section{ Основная часть}
     \newpage
     \subsection{Задание на курсовую работу}
     \upshape{\normalsize 1. Даны функции:}\\[4pt]
     {\,\,\,\,\,\,\,\,\,\,\,\,\,\,\,\,\,\,} {$f{(x)}$ = \sqrt{3sin(x) + cos(x)}, {\,}$и$ \\
     {g{(x)} = cos(2∙{x} + {\pi}/3)-1}\\[7pt]
    $ a)	Решить уравнение {$f{(x)}$ = ${g(x).}$ $\\
    $ b)	Исследовать функцию $h{(x)}$ = $f{(x)}$ – $g{(x)}$ на промежутке [0 ; (5∙\pi)/6].$\\ [10pt]
     \upshape{\normalsize 2. Найти коэффициенты кубического сплайна, интерполирующего данные, представленные в векторах $\vec{V}_x$ и $\vec{V}_y$. \\
     Построить на одном графике функцию $f(x)$ и функцию $f_1(x)$, полученную после нахождения коэффициентов кубического сплайна.\\
     Представить графическое изображение результатов интерполяции исходных данных различными методами с использованием встроенных функций $cspline(V_x,V_y)$, $pspline(V_x,V_y)$, $lspline(V_x,V_y)$ и $interp(V_k,V_x,V_y,x)$.}\\[4pt]
     \upshape{\normalsize 3. Решить задачу оптимального распределения неоднородных ресурсов.}\\[10pt]
     
     \begin{table}[ht]
     	\caption{Исходные данные}\label{tab:table_init}
     	\centering%центрируем таблицу	
     	 	
     	
     	\begin{tabular}{|1|1|1|1|1|1|}
     		\toprule
     		\hline
     			\multirow{2}{*}{Используемые ресурсы, $a_i$}&
     		\multicolumn{4}{|c|}{Изготавливаемые изделия}& 
     		\multirow{2}{c|}{Наличие ресурсов, $a_i$}\\
     		 		%\cmidrule{1-5}&
     		    		
     		\multicolumn{1}{|c|}{И_1}&
     		\multicolumn{1}{|c|}{И_2}&
     		\multicolumn{1}{|c|}{И_3}&
     		\multicolumn{1}{|c|}{И_4}&\\
     		 \hline
     		     		
     		\midrule 
     		Трудовые & 2 & 4 &  2 & 9  \\
     		\hline
     		Материальные & 5 & 5 &  5 & 6\\
     		\hline
     		Финансовые & 5 & 6 &  4 & 8\\
     		\hline
     		Прибыль & 25 & 45 & 60 & 20 \\ 
     		
     		\bottomrule
     		\hline
     	\end{tabular}
     	
     \end{table} 
     
     \newpage
     \subsection{Исследование и решение функций}
      \newpage
      \subsection{Нахождение коэффициентов кубического сплайна}
     \upshape
      \subsubsection{Задания и исходные данные для решения} 
      $ $1. Найти коэффициенты кубического сплайна, интерполирующего данные, представленные в векторах$\,\,  {\vec{V}_x} \,\,$и$\,\, {\vec{V}_y.}$ \\
      $ {\,\,\,\,\,\,\,\,\,\,\,\,\,\,\,\,\,\,}$2. Построить на одном графике: функцию$\,\, {f(x)}\,\, $и$\,\,  функцию {f_1(x)}, $полученную после нахождения коэффициентов кубического сплайна.$ $ \\
      $ {\,\,\,\,\,\,\,\,\,\,\,\,\,\,\,\,\,\,}$3. Представить графическое изображение результатов интерполяции исходных данных$ $.\\
                       
      $\vec{V}_x=\left(\begin{array}{c}0\\0.5\\1.4\\2.25\\3.5\end{array}\right),
      \,\,\,\vec{V}_y=\left(\begin{array}{c}3.0\\2.7\\3.7\\3.333\\3.667\end{array}\right)$ \\\\
      $ {\,\,\,\,\,\,\,\,\,\,\,\,\,\,\,\,\,\,}$Необходимо оценить погрешность в точке $ {x = 2.4}. $\,\,\,\,\,Вычислить значение функции в точке $\,{x = 1.2}.$\\
      \newpage
      \newpage
      \subsubsection{Теория и вывод уравнения сплайна}
      Уравнение сплайна находится по пяти точкам\\
      $(x_1;y_1), (x_2;y_2), (x_3;y_3), (x_4;y_4), (x_5;y_5)$\\
      Представим сплайн полиномом третьей степени на каждом отрезке
      $[x_i, x_{i+1}]$.
      
      \begin{equation}\label{eq:F_i(x)}
      F_i(x)=A_i0+{A_{i1}}x+{A_{i2}}x^2+{A_{i3}}x^3,
      \end{equation}
      
      $где $x$ \in {\,} $[{x_i},{x_{i+1}}].$\\[4pt]
      Найдем коэффициенты $A_{ij}$ исходя из того, что в точках склейки функция не имеет разрывов, изломов и изгиб ее слева и справа совпадает. \\
      На каждом из отрезков $[x_i, x_{i+1}]$ график $F_i(x)$ проходит через точки $y_i$, $y_{i+1}.$
      \begin{equation}\label{eq:y_i}
      y_i=A_{i0}+{A_{i1}}{x_i}+{A_{i2}}{x_i}^2+{A_{i3}}{x_i}^3
      \end{equation}
            
      Получаем $8$ уравнений:
      \begin{equation}\label{eq:y1(x)}
      \begin{aligned}
      y_1=A_{10}+{A_{11}}{x_1}+{A_{12}}{{x_1}^2}+{A_{13}}{x_1}^3\\[4pt]
      y_2=A_{10}+{A_{11}}{x_2}+{A_{12}}{x_2}^2+{A_{13}}{x_2}^3\\[4pt]
      y_2=A_{20}+{A_{21}}{x_2}+{A_{22}}{x_2}^2+{A_{23}}{x_2}^3\\[4pt]
      y_3=A_{20}+{A_{21}}{x_3}+{A_{22}}{x_3}^2+{A_{23}}{x_3}^3\\[4pt]
      y_3=A_{30}+{A_{31}}{x_3}+{A_{32}}{x_3}^2+{A_{33}}{x_3}^3\\[4pt]
      y_4=A_{30}+{A_{31}}{x_4}+{A_{32}}{x_4}^2+{A_{33}}{x_4}^3\\[4pt]
      y_4=A_{40}+{A_{41}}{x_4}+{A_{42}}{x_4}^2+{A_{43}}{x_4}^3\\[4pt]
      y_5=A_{40}+{A_{41}}{x_5}+{A_{42}}{x_5}^2+{A_{43}}{x_5}^3\\[4pt]
      \end{aligned}
      \end{equation}
       Производные первого порядка во внутренних точках ${x_i}$ должны совпадать,
       т.е. производная слева 
       $${{F_i}^{'}}({x_i}) = A_{i1}+ 2{A_{i2}}{x_i}+ 3A_{i3}{{x_i}^2}$$
        должна быть равна производной справа 
        $${{F^{'}}_{(i+1)}}({x_i}) = {A_{{(i+1)}1}}+ 2{A_{{(i+1)}2}}{x_i}+ 3A_{{(i+1)}3}{{x_i}^2}$$
      Физический смысл равенства производных состоит в том, что в точках склейки у нас нет излома сплайна.     
    \begin{equation}\label{eq:y^{'}(x)}
    \begin{aligned}
    A_{11}+ 2A_{12}{x_2}+ 3A_{13}{{x_2}^2}=A_{21}+ 2A_{22}{x_2}+ 3A_{23}{{x_2}^2}\\
    A_{21}+ 2A_{22}{x_3}+ 3A_{23}{{x_3}^2}=A_{31}+ 2A_{32}{x_3}+ 3A_{33}{{x_3}^2}\\
    A_{31}+ 2A_{32}{x_4}+ 3A_{33}{{x_4}^2}=A_{41}+ 2A_{42}{x_4}+ 3A_{43}{{x_4}^2}\\ 
    \end{aligned}
    \end{equation}
    
      Производные второго порядка в точках склейки ${x_i}$ должны совпадать,
      т.е. вторая производная слева 
      $${{F_i}^{''}}{(x_i)} = 2{A_{i2}}+ 6{A_{i3}}{x_i}$$
      должна быть равна второй производной справа 
      $${{F^{''}}_{(i+1)}{(x_i)} =2{A_{{(i+1)}2}+ 6{A_{{(i+1)}3}}{x_i}$$
         Физический смысл равенства вторых производных состоит в том, что в точках склейки изгиб сплайна справа и слева должен быть одинаковым.
            
      		\begin{equation}\label{eq:y^{'}(x)}
      		\begin{aligned}
      		2{A_{12}}+ 6{A_{13}}{x_2}=2{A_{22}}+ 6{A_{23}}{x_2}\\
      		2{A_{22}}+ 6{A_{23}}{x_3}=2{A_{32}}+ 6{A_{33}}{x_3}\\
      		2{A_{32}}+ 6{A_{33}}{x_4}=2{A_{42}}+ 6{A_{43}}{x_4}\\ 
      		\end{aligned}
      		\end{equation}
      		
      		Еще два уравнения - из граничных условий в крайних точках $x_1$, $x_n$:
      	\begin{equation}\label{eq:F^{''}(x)=0}
      	\begin{aligned}
      	{C_{11}}{F^{'}}{x_1}+{C_{12}}+ {{F^{''}}{(x_1)}={C_{13}}\\
      	C_{n1}{F^{'}}{n_1}+C_{n2}+ {{F^{''}}{(n_2)}=C_{n3}\\ 
      	\end{aligned}
      	\end{equation}
      						
      Найдем график сплайна в случае, когда концы сплайна оставлены
   	свободными в граничных точках $(x1, y1)$, $(x5, y5)$. Соответственно, уравнения имеют вид:
      	\begin{equation}\label{eq:F^{''}(x)}
      	\begin{aligned}
      	2A_{12}+ 6A_{13}{{x_1}=0\\
      	2A_{42}+ 6A_{43}{{x_5}=0\\
      	\end{aligned}
      	\end{equation}
      	В итоге - 16 уравнений для определения 16 коэффициэнтов $A_{ij}$ .	\\
      	
      		%\mathds{A} = \left(
      	\\
      		{\tiny
      		
      		\left(\begin{array}{cccccccccccccccc} 
      			1&{x_1}&{x_1}^2&{x_1}^3&0&0&0&0&0&0&0&0&0&0&0&0\\
      			1&{x_2}&{x_2}^2&{x_2}^3&0&0&0&0&0&0&0&0&0&0&0&0\\
      			0&1&2{x_2}&3{x_1}^2&0&-1&-2{x_2}&-3{x_2}^2&0&0&0&0&0&0&0&0\\
      		    0&0&2&6{x_2}&0&0&-2&-6{x_2}&0&0&0&0&0&0&0&0\\
      		    0&0&0&0&1&{x_2}&{x_2}^2&{x_2}^3&0&0&0&0&0&0&0&0\\
      		    0&0&0&0&1&{x_3}&{x_3}^2&{x_3}^3&0&0&0&0&0&0&0&0\\
      		    0&0&0&0&0&1&2{x_3}&3{x_3}^2&0&-1&-2{x_3}&-3{x_3}^2&0&0&0&0\\
      		    0&0&0&0&0&0&2&6{x_3}&0&0&-2&-6{x_3}&0&0&0&0\\
      		    0&0&0&0&0&0&0&0&1&{x_3}&{x_3}^2&{x_3}^3&0&0&0&0\\
      		    0&0&0&0&0&0&0&0&1&{x_4}&{x_4}^2&{x_4}^3&0&0&0&0\\
      		    0&0&0&0&0&0&0&0&0&1&2{x_4}&3{x_4}^2&0&-1&-2{x_4}&-3{x_4}^2\\
      		    0&0&0&0&0&0&0&0&0&0&2&6{x_4}&0&0&-2&-6{x_4}\\
      		    0&0&0&0&0&0&0&0&0&0&0&0&1&{x_4}&{x_4}^2&{x_4}^3\\
      		    0&0&0&0&0&0&0&0&0&0&0&0&1&{x_5}&{x_5}^2&{x_5}^3\\
      		    0&0&2&6{x_1}&0&0&0&0&0&0&0&0&0&0&0&0\\
      		    0&0&0&0&0&0&0&0&0&0&0&0&0&0&2&6{x_5}
      		    \end{array}\right)
      	$x$
      	\left(\begin{array}{c} 
      		$A_{10}$\\$A_{11}$\\$A_{12}$\\	$A_{13}$\\	
      		$A_{20}$\\$A_{21}$\\$A_{22}$\\	$A_{23}$\\ 
      		$A_{30}$\\$A_{31}$\\$A_{32}$\\	$A_{33}$\\ 
      		$A_{40}$\\$A_{41}$\\$A_{42}$\\	$A_{43}$
      	\end{array}\right)
      		$=$
      		\left(\begin{array}{c} 
      			${y_{1}}$\\$y_{2}$\\$0$\\	$0$\\	
      			$y_{2}$\\$y_{3}$\\$0$\\	$0$\\ 
      			$y_{3}$\\$y_{4}$\\$0$\\	$0$\\ 
      			$y_{4}$\\$y_{5}$\\$0$\\	$0$\\ [15pt]
      		  	\end{array}\right)						
      	
      }\\
     \\[10pt]
      	
      	\centering{\normalsize Коэффициенты $A_{ij}:$}\\
      	              
      $$\centering{\normalsize  
      	\begin{vmatrix}
      \large A_{10}\\A_{11}\\ A_{12}\\ A_{13}\\
      A_{20}\\A_{21}\\A_{22}\\A_{23}\\
      A_{30}\\A_{31}\\A_{32}\\A_{33}\\
      A_{40}\\A_{41}\\A_{42}\\A_{43}
      \end{vmatrix}
      \centering{\large =}
       \begin{vmatrix}
       3\\-1.458\\ 0\\ 3.43\\
       4.43\\-8.49\\12.53\\-4.9\\
       -10.8\\36.23\\-19.4\\2.7\\
       20.65\\-5.7\\-0.735\\-0.07
       \end{vmatrix}}$$
       \newpage
       
       {\left \normalsize $Уравнение сплайна имеет вид:$}\\[10pt]
     {\normalsize ${F(x)}$= \left\{ 
      	\begin{aligned}
      	F_1(x)&=3.0x^3-1.46x^2+0.0x+3.43=0,\,\, $где$ \,\,\,{x}\in\,\,\, {$[0,\,0.5]$}};\\
      F_2(x)&=4.43x^3-8.49x^2+12.53x-4.9=0,\,\, $где$ \,\,\,{x}\in\,\,\, {$[0.5,\,1.4]$}};\\
    F_3(x)&=-10.8x^3-36.23x^2-19.4x+2.7=0,\,\, $где$ \,\,\,{x}\in\,\,\, {$[1.4,\,2.25]$}};\\
F_4(x)&=20.65x^3-5.7x^2-0.735x-0.07=0,\,\, $где$ \,\,\,{x}\in\,\,\, {$[2.25,\,3.5]$}}\\
      	\end{aligned}
      	
      
      	\right.\]
    }
      \newpage
      \subsection{Решение задачи оптимального распределения неоднородных ресурсов}
      \newpage
      	\section{ Заключение}
\end{document}